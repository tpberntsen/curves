\documentclass{article}
\usepackage{amsmath}
\usepackage[round]{natbib}
\usepackage{url}
\usepackage{hyperref}
\usepackage[toc,page]{appendix}
\setcounter{MaxMatrixCols}{20}
\DeclareMathOperator{\csch}{csch}

\title{Tension Spline Algorithm for Building Commodity Forward Curves}
\author{Jake C. Fowler}
\date{}

\begin{document}
\newcommand{\+}[1]{\ensuremath{\mathbf{#1}}}

\maketitle

THIS DOCUMENT IS CURRENTLY WORK IN PROGRESS

\section{Introduction}
\subsection{Reason for Interpolating Commodity Forward Curves}
The aim of a curve interpolation routine is to take a collection of traded forward prices, and tranform these into a forward curve
of homogenous granularity. Typically the derived curve can constructed to be in 
a granularity higher than what is traded in the market. But why would anyone want
to derive interpolated forward prices given that we cannot actually trade at these
prices? Often physical positions and structured deals will have a volume profile
which is at a higher granularity than forward market prices. For example natural
gas storage which can be withdrawn quickly over one or two weeks. If such a deal
were to be valued using traded contracts, and the Q1 price is the highest granularity
of such traded contracts, then such a deal would be undervalued.

\bigskip

The interpolated forward curve should consist of prices, prior to bid-offer and
any other adjustment, at which we are willing to mark positions and execute trades
for volume profiles at higher granularity to the traded forward market.

Next we need to think about the functional form of this interpolation. 

TODO: example of why we don't want to interpolate piecewise flat.
TODO: replicate high-granularity positions by rolling forward price hedges.

\section{Deriving the Algorithm}
\subsection{Functional Form}
The base of the algorithm is a spline, which by definition is made up of piecewise polynomial functions.
\begin{equation}
p(t) = 
\begin{cases}
    p_1(t)\qquad \text{for}\ \quad t \in [t_0, t_1) \\
    p_2(t)\qquad \text{for}\ \quad t \in [t_1, t_2) \\
    \;\vdots \\
    p_{n-1}(t)\quad \text{for}\ \quad t \in [t_{n-2}, t_{n-1}) \\
    p_n(t)\qquad \text{for}\ \quad t \in [t_{n-1}, t_n]
\end{cases}
\end{equation}
Where $t_0 < t_1 < \hdots < t_{n-1} < t_{n}$ are the boundary points between the polynomials which make up the spline.
In the context of building a forward curve, the variable $t$ is defined as
the time until start of delivery of a forward contract.

\bigskip

The boundary points are chosen to be start of the input forward prices. It is also
assumed that the input forward prices are not for delivery periods which overlap
with any other input. Gaps between input forward contracts are permitted, in which
case a boundary point will exit for the start of the gap.

\bigskip

\begin{multline}
p_i(t) = \frac{z_{i-1} \sinh(\tau_i (t_i - t)) + z_i \sinh(\tau_i (t - t_{i-1}))}{\tau_i^2 \sinh(\tau h_i)}  \\
    + \frac{(y_{i-1} - z_{i-1}/\tau_i^2)(t_i - t) + (y_i - z_i/\tau_i^2)(t - t_{i-1})}{h_i}
\end{multline}

Where $h_i =  t_i - t_{i-1}$. 
$z_i = p^{\prime \prime}(t_i)$ and $y_i = p(t_i)$, i.e. the (as yet unknown) value of the function at
the boundary points.

The curve fitting algorithm essentially involves solving for the parameters
$z_i$, and $y_i$ for $i=0 \hdots n$.

\bigskip
In many cases the spline described above is not sufficient to derive a forward curve which
shows strong price seasonality, especially when this seasonality cannot be directly observed
in the traded forward prices. An example of this is the day-of-week seasonality for gas and
power prices, which generally are lower at the weekend when demand is lower. As such the 
function form is as follows:

\begin{equation}
    \label{eq:foward_function}
    f(t) = (p(t) + S_{add}(t))S_{mult}(t)
\end{equation}

Where the forward price for the period starting delivery at time $t$ is given by $f(t)$, which 
consists of $p(t)$ adjusted by two arbitrary seasonal adjustment functions
$S_{add}(t)$ an additive adjustment, and $S_{mult}(t)$ a multiplicative adjustment.


\subsection{Constraints}
\subsubsection{Polynomial Boundary Point Constraints}
As usual with splines, constraints are put in place that adjascent polynomials 
have equal value, first derivative, and second derivatives at the boundary points.

\subsubsection{Polynomial Value Boundary Point Equality}
To make $p(t)$ continuous we need to constrain $p_i(t_{i-1}) = p_{i-1}(t_{i-1})$. Evaluating both
of these:

\begin{eqnarray}
    \nonumber
    p_i(t_{i-1}) = \frac{z_{i-1} \sinh(\tau_i h_i) + z_i \sinh(0)}{\tau_i^2 \sinh(\tau_i h_i)}
    + \frac{(y_{i-1} - z_{i-1}/\tau_i^2) h_i }{h_i} \\
    \nonumber
    = \frac{z_{i-1}}{\tau_i^2} + y_{i-1} - \frac{z_{i-1}}{\tau_i^2} \\
    = y_{i-1}
\end{eqnarray}

\begin{eqnarray}
    \nonumber
    p_{i-1}(t_{i-1}) = \frac{z_{i-2} \sinh(0) + z_{i-1} \sinh(\tau_{i-1} h_{i-1})}{\tau_{i-1}^2 \sinh(\tau_{i-1} h_{i-1})}
    + \frac{(y_{i-1} - z_{i-1}/\tau_{i-1}^2) h_{i-1}}{h_{i-1}} \\
    \nonumber
    = \frac{z_{i-1}}{\tau_{i-1}^2} + y_{i-1} - \frac{z_{i-1}}{\tau_{i-1}^2} \\
    \nonumber
    = y_{i-1}
\end{eqnarray}

Hence, by construction, $p(t)$ is always continuous with value $y_{i-1}$ at the boundary
between $p_i$ and $p_{i - 1}$

\subsubsection{Polynomial First Derivative Boundary Point Equality}
This is to constrain $p^\prime_i(t_{i-1}) = p^\prime_{i-1}(t_{i-1})$. First finding the 
expression for $p^\prime_i(t)$:

\begin{multline}
    p_i(t) = \frac{z_{i-1} \sinh(\tau_i (t_i - t)) + z_i \sinh(\tau_i (t - t_{i-1}))}{\tau_i^2 \sinh(\tau h_i)}  \\
        + \frac{(y_{i-1} - z_{i-1}/\tau_i^2)(t_i - t) + (y_i - z_i/\tau_i^2)(t - t_{i-1})}{h_i}
\end{multline}

\begin{multline}
    p^\prime_i(t) = \frac{ -z_{i-1} \cosh(\tau_i (t_i - t)) + z_i \cosh(\tau_i (t - t_{i-1}))}{\tau_i \sinh(\tau_i h_i)}  \\
        + \frac{y_i - y_{i-1} +  (z_{i-1} - z_i)/\tau_i^2}{h_i}
\end{multline}

For clarity, rearranging this to highlight the linearity with respect to the parameters:

\begin{multline}
    p^\prime_i(t) = z_i \biggl( \frac{ \cosh(\tau_i (t - t_{i-1}))}{\tau_i \sinh(\tau_i h_i)} - \frac{1}{h_i \tau_i^2} \biggr) 
        + z_{i-1} \biggl( \frac{1}{h_i \tau_i^2} - \frac{ \cosh(\tau_i (t_i - t))}{\tau_i \sinh(\tau_i h_i)} \biggr)\\
        + y_i \frac{1}{h_i} - y_{i - 1} \frac{1}{h_i}
\end{multline}

Evaluating this about the boundary points:

\begin{multline}
    p^\prime_i(t_{i-1}) = z_i \biggl( \frac{1}{\tau_i \sinh(\tau_i h_i)} - \frac{1}{h_i \tau_i^2} \biggr) 
        + z_{i-1} \biggl( \frac{1}{h_i \tau_i^2} - \frac{ \cosh(\tau_i h_i)}{\tau_i \sinh(\tau_i h_i)} \biggr)\\
        + y_i \frac{1}{h_i} - y_{i - 1} \frac{1}{h_i}
\end{multline}

\begin{multline}
    p^\prime_{i-1}(t_{i-1}) = z_{i-1} \biggl( \frac{ \cosh(\tau_{i-1} h_{i-1})}{\tau_{i-1} \sinh(\tau_{i-1} h_{i-1})} - \frac{1}{h_{i-1} \tau_{i-1}^2} \biggr) 
        + z_{i-2} \biggl( \frac{1}{h_{i-1} \tau_{i-1}^2} - \frac{1}{\tau_{i-1} \sinh(\tau_{i-1} h_{i-1})} \biggr)\\
        + y_{i-1} \frac{1}{h_{i-1}} - y_{i-2} \frac{1}{h_{i-1}}
\end{multline}

Setting these equal:

\begin{multline}
    0 = z_i \biggl( \frac{1}{\tau_i \sinh(\tau_i h_i)} - \frac{1}{h_i \tau_i^2} \biggr) \\
        + z_{i-1} \biggl( \frac{1}{h_i \tau_i^2} -\frac{ \cosh(\tau_i h_i)}{\tau_i \sinh(\tau_i h_i)}
        - \frac{ \cosh(\tau_{i-1} h_{i-1})}{\tau_{i-1} \sinh(\tau_{i-1} h_{i-1})} + \frac{1}{h_{i-1} \tau_{i-1}^2}\biggr)\\
        - z_{i-2} \biggl( \frac{1}{h_{i-1} \tau_{i-1}^2} - \frac{1}{\tau_{i-1} \sinh(\tau_{i-1} h_{i-1})} \biggr)\\
        + y_i \frac{1}{h_i} - y_{i - 1} \bigl( \frac{1}{h_i} + \frac{1}{h_{i-1}} \bigr)
        + y_{i-2} \frac{1}{h_{i-1}}
\end{multline}

This constraint should be held for $i = 2 \hdots n$.

\bigskip

The above three equation should hold for the boundary points \\
$t \in \{t_1, t_2, \hdots, t_{n-2}, t_{n-1}\}$.

\subsubsection{Forward Price Constraint}
The most important constraints is that the derived forward curve averages back to the 
input traded forward prices.
The market inputs to the forward curve model are traded forward prices $F_i$.
Setting this equal to the average of the derived smooth curve:

\begin{equation}
    \label{eq:traded_forward_calc}
    F_j = \frac{\sum_{t \in T_j} (p(t) + S_{add}(t))S_{mult}(t)w(t)D(t)}
    {\sum_{t \in T_j} w(t)D(t)}
\end{equation}

Where $D(t)$ is the discount factor from the settlement date of delivery period $t$.
$w(t)$ is a weighting function and $T_i$ is the set of all delivery start times
for the delivery periods at the granularity of the curve being built. The weighting function has two
meanings from a busines perspective.
\begin{itemize}
    \item The volume of commodity delivered in each period. For example, an off-peak power forward 
    contract in the UK delivers over 12 hours in on weekdays, and 24 hours on weekends, hence $w(t)$ 
    would equal double for $t$ representing weekends compared to $w(t)$ when $t$ represents a weekday delivery. 
    Clock changes can also cause the total volume delivered over a day in a fixed time zone to vary 
    due to hours lost or gained. Hence $w(t)$ can be used to account for this.
    \item For swaps which only fix on certain days (usually business days) $w(t)$ can be used to account 
    for this by returning the number of fixing days in the period starting at $t$.
    For example if deriving a a monthly curve $w(t)$ would evaluate to the number of fixing days in the 
    month starting at $t$.
\end{itemize}
Equation \ref{eq:traded_forward_calc} can be transformed into an equation linear on the parameters of the piecewise polynomial by substituting
in the polynomial representation of $p(t)$:

\begin{equation}
    \sum_i \sum_{t \in T_j \cap  [t_{i-1}, t_i) } p_i(t) S_{mult}(t)w(t)D(t) = 
    F_j \sum_{t \in T_j} w(t) D(t) - \sum_{t \in T_i} S_{add}(t) S_{mult}(t)w(t)D(t)
\end{equation}



Substituting in for $p_i(t)$:
\begin{multline}
    \sum_i \sum_{t \in T_i \cap  [t_{i-1}, t_i)} \biggl( \frac{z_{i-1} \sinh(\tau_i (t_i - t)) + z_i \sinh(\tau_i (t - t_{i-1}))}{\tau_i^2 \sinh(\tau_i h_i)}  \\
    + \frac{(y_{i-1} - z_{i-1}/\tau_i^2)(t_i - t) + (y_i - z_i/\tau_i^2)(t - t_{i-1})}{h_i} + 
    S_{add}(t) \biggr) S_{mult}(t)w(t)D(t) \\
    = F_i \sum_{t \in T_i} w(t)D(t) - \sum_{t \in T_i} S_{add}(t) S_{mult}(t)w(t)D(t)
\end{multline}


Rearranging again gives a form linear with respect to the unknown polynomial coefficients
$z_i$, $z_{-1}$, $y_i$ and $y_{i-1}$.

\begin{multline}
    \sum_i \biggl(  z_i \sum_{t \in T_j \cap [t_{i-1}, t_i)} \biggl( \frac{\sinh(\tau_i (t - t_{i-1}))}{\tau_i^2 \sinh(\tau_i h_i)} 
      - \frac{t - t_{i-1}}{\tau_i^2 h_i} \biggr) S_{mult}(t)w(t)D(t) \\
    + z_{i-1} \sum_{t \in T_j \cap [t_{i-1}, t_i)} \biggl( \frac{\sinh(\tau_i (t_i - t))}{\tau_i^2 \sinh(\tau_i h_i)} 
     - \frac{t_i - t}{\tau_i^2 h_i} \biggr) S_{mult}(t)w(t)D(t) \\
    + y_i \sum_{t \in T_j \cap [t_{i-1}, t_i)} \frac{(t - t_{i-1})}{h_i} S_{mult}(t)w(t)D(t) \\
    + y_{i-1} \sum_{t \in T_j \cap [t_{i-1}, t_i)} \frac{(t_i - t)}{h_i} S_{mult}(t)w(t)D(t) \biggr) \\
    = F_j \sum_{t \in T_i} w(t)D(t) - \sum_{t \in T_i} S_{add}(t) S_{mult}(t)w(t)D(t)
\end{multline}

This constraint should be held for $j=1 \hdots n$.

\subsubsection{Matrix Form of Constraints}
%Equations \ref{eq:continuity_constraint}, \ref{eq:1st_deriv_constraint}, \ref{eq:2nd_deriv_constraint} and
% \ref{eq:price_constraint} can be expressed as the linear system $\+{Ax = b}$ where:


Start with forward price constraint as less lags (probably)

\begin{equation}
    \alpha^j_i = 
\end{equation}

Superscript is for contract, supersci

\begin{equation*}
    \begin{bmatrix}
        1 & 0 & 0 & 0 & 0 & 0 & \hdots & 0 & 0 & 0 & 0 & 0 & 0 \\
        \alpha^1_0 & \beta^1_0 & \gamma^1_0 & \delta^1_0 & \alpha^1_1 & \beta^1_1 & \hdots & 0 & 0 & \alpha^1_n & \beta^1_n & \gamma^1_n & \delta^1_n \\
        0 & 0 & 0 & 0 & 0 & 0 & \hdots & 0 & 0 & 0 & 0 & 0 & 0 \\
        \alpha^2_0 & \beta^2_0 & \gamma^2_0 & \delta^2_0 & \alpha^2_1 & \beta^2_1 & \hdots & 0 & 0 & \alpha^2_n & \beta^2_n & \gamma^2_n & \delta^2_n \\        
        \vdots & & & & & & \ddots & & & & & & \vdots \\
        0 & 0 & 0 & 0 & 0 & 0 & \hdots & 0 & 0 & 0 & 0 & 0 & 0 \\
        \alpha^n_0 & \beta^n_0 & \gamma^n_0 & \delta^n_0 & \alpha^n_1 & \beta^n_1 & \hdots & 0 & 0 & \alpha^n_n & \beta^n_n & \gamma^n_n & \delta^n_n \\
        0 & 0 & 0 & 0 & 0 & 0 & \hdots & 0 & 0 & 0 & 0 & 0 & 0 \\
        0 & 0 & 0 & 0 & 0 & 0 & \hdots & 0 & 0 & 0 & 0 & 1 & 0 
    \end{bmatrix}
    \begin{bmatrix}
        z_0 \\
        y_0 \\
        z_1 \\
        y_1 \\
        \vdots \\
        z_{n-1} \\
        y_{n-1} \\
        z_n \\
        y_n \\
    \end{bmatrix} =
    \begin{bmatrix}
        0 \\
        f_i \\
        0 \\
        0 \\
        \vdots \\
        0 \\
        0 \\
        0 \\
        0 \\
    \end{bmatrix}
\end{equation*}

\subsection{Smoothness Criteria}
% TODO add reference to Andersen paper
We want to find the solution which minimises the following.
\begin{equation}
    \int_{t_0}^{t_n} \bigl( p^{\prime\prime}(t)^2 + \tau_i^2 p^{\prime}(t)^2 \bigr) dt = 
    \sum_{i=1}^{n}\int_{t_{i - 1}}^{t_i}p^{\prime\prime}_i(t)^2 + \tau_i^2 p^{\prime}(t)^2 dt
\end{equation}

The integral over $p^{\prime\prime}(t)^2$ penalises the change in curve direction
seen in oscillations. The $p^{\prime}(t)^2$ term penalises total curve length
of oscillations. A reason for this being weighted by $\tau_i^2 $ can be seen as follows.
In the case of high tension parameter, the pieceswise function will be virtually linear,
hence $p^{\prime\prime}(t)^2$ will be tiny, so there is no point in penalising for this.
Rather, we would rather penalise for the amplitute of
zigzagging which can be seen when interpolating with a piecewise linear function.

\bigskip

Note that the final forward curve is actually read off discrete points from the piecewise
function, hence the penalty function is more appropriately represented as a summation
of $p^{\prime\prime}(t)^2 + \tau_i^2 p^{\prime}(t)^2$ over these discrete points, rather 
than an integral. However, as shown below, the integral 
can be evaluated to a convenient form, and given we are usually interpolating to a high
granularity, the spacing of the discrete points will be small, so the integral will be 
a close approximation of a summation.

\bigskip

See Appendix \ref{appendix:max_smooth_integral} for the evaluation of this integral to the
following.

\begin{multline}
    = z_i^2 \biggl( \frac{\cosh(\tau_i h_i)}{\tau_i \sinh(\tau_i h_i)} -
    \frac{1}{\tau_i^2 h_i} \biggr) 
    + z_{i-1}^2 \biggl( \frac{\cosh(\tau_i h_i)}{\tau_i \sinh(\tau_i h_i)} -
    \frac{1}{\tau_i^2 h_i} \biggr) \\
    + y_i^2 \frac{\tau_i^2}{h_i} + y_{i-1}^2 \frac{\tau_i^2}{h_i} 
    + z_i z_{i-1} 2 \biggl( \frac{1}{\tau_i^2 h_i} - \frac{1}{\tau_i \sinh(\tau_i h_i)} \biggr) \\
    - y_i y_{i-1} \frac{2 \tau_i^2}{h_i}
\end{multline}

Writing this in symmetric matrix form.

\begin{equation}
    = \sum_{i=1}^{n} \begin{bmatrix}
        z_{i-1} \\ y_{i-1} \\ z_i \\ y_i
    \end{bmatrix}^T
    \begin{bmatrix}
        \left(\frac{\cosh(\tau_i h_i)}{\tau_i \sinh(\tau_i h_i)} - \frac{1}{\tau_i^2 h_i} \right) & 0 & \left( \frac{1}{\tau_i^2 h_i} - \frac{1}{\tau_i \sinh(\tau_i h_i)} \right) & 0 \\
        0 & \frac{\tau_i^2}{h_i} & 0 & -\frac{\tau_i^2}{h_i} \\
        \left(\frac{1}{\tau_i^2 h_i} - \frac{1}{\tau_i \sinh(\tau_i h_i)}\right) & 0 & \left(\frac{\cosh(\tau_i h_i)}{\tau_i \sinh(\tau_i h_i)} - \frac{1}{\tau_i^2 h_i} \right) & 0 \\
        0 & -\frac{\tau_i^2}{h_i} & 0 & \frac{\tau_i^2}{h_i} \\
    \end{bmatrix}
    \begin{bmatrix}
        z_{i-1} \\ y_{i-1} \\ z_i \\ y_i
    \end{bmatrix}
\end{equation}

This can be changed from a sum to a single quadratic form.

\begin{equation}
    = \begin{bmatrix}
        z_0 \\
        y_0 \\
        z_1 \\
        y_1 \\
        \vdots \\
        z_{n-1} \\
        y_{n-1} \\
        z_n \\
        y_n \\
    \end{bmatrix} ^T
    \+H   
    \begin{bmatrix}
        z_0 \\
        y_0 \\
        z_1 \\
        y_1 \\
        \vdots \\
        z_{n-1} \\
        y_{n-1} \\
        z_n \\
        y_n \\
    \end{bmatrix}
\end{equation}

\begin{equation}
    \+H = \begin{bmatrix}
        \left(\frac{\cosh(\tau_1 h_1)}{\tau_1 \sinh(\tau_1 h_1)} - \frac{1}{\tau_1^2 h_1} \right) & 0 & \left( \frac{1}{\tau_1^2 h_1} - \frac{1}{\tau_1 \sinh(\tau_1 h_1)} \right) & 0 & 0 & 0 & \hdots & 0 & 0 & 0 & 0 & 0 & 0 \\
        0 & \frac{\tau_1^2}{h_1} & 0 & -\frac{\tau_1^2}{h_1} & 0 & 0 & \hdots & 0 & 0 & 0 & 0 & 0 & 0 \\
        \left( \frac{1}{\tau_1^2 h_1} - \frac{1}{\tau_1 \sinh(\tau_1 h_1)} \right) & 0 & \left(\frac{\cosh(\tau_1 h_1)}{\tau_1 \sinh(\tau_1 h_1)} - \frac{1}{\tau_1^2 h_1} \right) + \left(\frac{\cosh(\tau_2 h_2)}{\tau_2 \sinh(\tau_2 h_2)} - \frac{1}{\tau_2^2 h_2} \right) & 0 & 0 & 0 & \hdots & 0 & 0 & 0 & 0 & 0 & 0 \\
        0 & -\frac{\tau_1^2}{h_1}  & 0 & \frac{\tau_1^2}{h_1} + \frac{\tau_2^2}{h_2} & 0 & 0 & \hdots & 0 & 0 & 0 & 0 & 0 & 0 \\        
        \vdots & & & & & & \ddots & & & & & & \vdots \\
        0 & 0 & 0 & 0 & 0 & 0 & \hdots & 0 & 0 & 0 & 0 & 0 & 0 \\
        0 & 0 & 0 & 0 & 0 & 0 & \hdots & 0 & 0 & 0 & 0 & 0 & 0 \\
        0 & 0 & 0 & 0 & 0 & 0 & \hdots & 0 & 0 & 0 & 0 & 0 & 0 \\
        0 & 0 & 0 & 0 & 0 & 0 & \hdots & 0 & 0 & 0 & 0 & 0 & 0 
    \end{bmatrix}   
\end{equation}

\section{Future Work}
\subsection{Collinear Inputs}
Any redundancy seen in the input market data (e.g. contracts for Q-1, Jan, Feb and Mar delivery all present)
will result in matrix TODO above being singular, hence the linear system cannot be solved.
Theoretically redundant input forward prices should be consistent and arbitrage free.
However, real-world market frictions such as bid-offer spread, transaction costs and limited
liquidity mean that small theoretically arbitragable inconsistencies between collinear forward
prices are likely to be seen.

If the curves are built using a highly automated curve building tool this will be a problem.
This section presents some ideas on how to tackle this.

\subsubsection{Preprocess Input Contracts}
It could make sense to discard a suffient set of input contracts 
such that the linear system is solvable. Firstly the criteria for which contracts
are kept needs to be defined. As we are looking to build a curve with increased granularity
this could be to prioritise contracts with shorter delivery period. Alternatively, the
criteria could be based on liquidity.

Once the criteria has been defined, the input forward prices and shaping factors would
be ordered, starting with the most favourable to be retained. The inputs would then
be looped through, a matrix constructed containing the volume profiles, and then the rank
of this matrix calculated. If the addition of any input does not increase the rank of
the volume matrix, then this input is collinear with inputs that have more favourable 
selection criteria, hence should be discarded. Some additional thought would likely
yield a more efficient algorithm which does not involve fully recalculating the rank 
after the addition of each input contract. For example a rank calculation which makes
use of intermediate results in prior rank calculations.

\bigskip

One disadvantage of this approach is that market price information is being lost. The
criteria for discarding contracts could lead prices which aren't good
representations of the actual liquid prices being used.

\subsubsection{Numerical Minimisation}
An approach which uses and retains all input market data will need to account
for the fact that the interpolated curve will not exactly average back to all input
forward prices. The obvious approach is to solve for a curve which minimises the
weighted sum of squared differences between input forward prices and the averaged 
interpolated curve. However, this problem is complicated by the linear system
already having rank less than the number of unknowns being solved for, which is why
the maximum smoothness solution is choosen from the infinite available solutions.
We need to find an objective function which accounts for minimising both the squared
difference to inputs contracts and maximises smoothness by minimising integral 
TODO equation reference. A simple objective function would be the weighted average
of these two terms. A numerical minimisation routine would then be used to find the
spline parameters which minimised this objective function. It's likely that the
derivative of this objective function has an analytical form, hence a form of
gradient descent could be used.

\bigskip

One problem with this approach is that the weighting in the objetive function
needs to be arbitrarily specified to determine the relative importance of smoothness
and closeness to input contract prices.

\bigskip

Another complication is that that the first derivative continuity constraints must
be exactly met for a solution to be acceptable. This leaves us in the unusual situation
of having a linear system where some constraints must be exactly met, whilst for other
linear constraints we are happy with the least squares solution. As a starting point
to solve this problem, the set of coefficient vectors which exactly obey the first 
derivative continuity can be found by finding the nullspace of a matrix containing
these constraints.

\subsubsection{Two Step Solution}
To avoid the arbitrary weigting factor in the objective function in the approach 
described above one could consider that minimising the squared difference of interpolated
curve to the input forward prices is more important than maximising the smoothness.
A two-step solution would be to first find the general solution form (arbitrary solution
and basis of the nullspace) which minimised the squared difference to market
prices using an SVD. As a second step, a specific solution (as coefficients of the 
nullspace basis) which then maximises the smoothness is then solved for.

\subsection{Alternative Tension Spline}
There are alternative tension splines to the hyperbolic spline used in this 
document, for example the Cardinal Spline. The hyperbolic tension spline was chosen
over the Cardinal Spline for no particularly reason. It is likely that there are pros
and cons of the hyperbolic and Cardinal splines over each other for use in constructing commodity
forward curves. A future project would be to implement a similar algorithm to the one
in this document, but using the Cardinal Spline, and comparing the practicalities of each approach.

\newpage
\appendix
\appendixpage
\section{Maximum Smoothness Integral}
\label{appendix:max_smooth_integral}
This section evaluated the integral used in the maximum smoothness criteria. First writing
the squares as multiples, and splitting the integral into two:

\begin{equation}
    \int_{t_{i - 1}}^{t_i}p^{\prime\prime}_i(t)^2 + \tau_i^2 p^{\prime}_i(t)^2 dt
    = \int_{t_{i - 1}}^{t_i}p^{\prime\prime}_i(t) p^{\prime\prime}_i(t) dt +
    \int_{t_{i - 1}}^{t_i} \tau_i^2 p^{\prime}_i(t) p^{\prime}_i(t) dt
\end{equation}

The first integral on the right-hand-side can be evaluated using integration by parts.

\begin{equation}
    \int_{t_{i - 1}}^{t_i}p^{\prime\prime}_i(t) p^{\prime\prime}_i(t) dt = 
    p^{\prime\prime}_i(t_i) p^{\prime}_i(t_i) - p^{\prime\prime}_i(t_{i - 1}) p^{\prime}_i(t_{i - 1})
    - \int_{t_{i - 1}}^{t_i} p^{\prime\prime\prime}_i(t) p^{\prime}_i(t_{i - 1}) dt
\end{equation}

Substituing this into the first equation in this appendix and combining the last two integrals
into a single integral.

\begin{equation}
    \int_{t_{i - 1}}^{t_i}p^{\prime\prime}_i(t)^2 + \tau_i^2 p^{\prime}(t)^2 dt
    = p^{\prime\prime}_i(t_i) p^{\prime}_i(t_i) - p^{\prime\prime}_i(t_{i - 1}) p^{\prime}_i(t_{i - 1})
    - \int_{t_{i - 1}}^{t_i} \bigl( p^{\prime\prime\prime}_i(t) - \tau_i^2 p^{\prime}_i(t) \bigr) p^{\prime}_i(t) dt
\end{equation}

Next, focussing on the $p^{\prime\prime\prime}_i(t) - \tau_i^2 p^{\prime}_i(t)$ term. It has previously been
shown that the first derivative is as follows:

\begin{multline}
    p^\prime_i(t) = \frac{ -z_{i-1} \cosh(\tau_i (t_i - t)) + z_i \cosh(\tau_i (t - t_{i-1}))}{\tau_i \sinh(\tau_i h_i)}  \\
        + \frac{y_i - y_{i-1} +  (z_{i-1} - z_i)/\tau_i^2}{h_i}
\end{multline}

Differentiating:

\begin{equation}
    p^{\prime\prime}_i(t) = \frac{ z_{i-1} \sinh(\tau_i (t_i - t)) + z_i \sinh(\tau_i (t - t_{i-1}))}{\sinh(\tau_i h_i)}
\end{equation}

Differentiating again:

\begin{equation}
    p^{\prime\prime\prime}_i(t) = \frac{ -z_{i-1} \tau_i \cosh(\tau_i (t_i - t)) + z_i \tau_i \cosh(\tau_i (t - t_{i-1}))}{\sinh(\tau_i h_i)}
\end{equation}

Using the above it can be seen that:

\begin{equation}
    p^{\prime\prime\prime}_i(t) - \tau_i^2 p^{\prime}_i(t) =
    - \frac{(y_i - y_{i-1})\tau_i^2 +  z_{i-1} - z_i}{h_i}
\end{equation}

Crucially this is independent of $t$, hence can be used to simply the equation.

\begin{multline}
    \int_{t_{i - 1}}^{t_i} \bigl( p^{\prime\prime\prime}_i(t) - \tau_i^2 p^{\prime}_i(t) \bigr) p^{\prime}_i(t) dt
    = - \frac{(y_i - y_{i-1})\tau_i^2 +  z_{i-1} - z_i}{h_i} \int_{t_{i - 1}}^{t_i} p^{\prime}_i(t) dt \\
    = - \frac{(y_i - y_{i-1})\tau_i^2 +  z_{i-1} - z_i}{h_i} \bigl(p_i(t_i) - p_i(t_{i-1}) \bigr)
\end{multline}

Putting these results together.

\begin{equation}
    \int_{t_{i - 1}}^{t_i}p^{\prime\prime}_i(t)^2 + \tau_i^2 p^{\prime}(t)^2 dt
    = p^{\prime\prime}_i(t_i) p^{\prime}_i(t_i) - p^{\prime\prime}_i(t_{i - 1}) p^{\prime}_i(t_{i - 1})
    + \frac{(y_i - y_{i-1})\tau_i^2 +  z_{i-1} - z_i}{h_i} \bigl(p_i(t_i) - p_i(t_{i-1}) \bigr)
\end{equation}

Evaluating the above involves substituting in for $p_i$ and it's first two derivatives at the integral
boundary points $t_i$ and $t_{i-1}$. By construction we already know $p_i(t_i) = y_i$, $p_i(t_{i-1}) = y_{i-1}$,
$p^{\prime\prime}_i(t_i) = z_i$ and $p^{\prime\prime}_{i-1}(t_{i-1}) = z_{i-1}$. The expressions for the first
derivatives can be simplied by subsituting $\cosh(0)=1$ and $t_i - t_{i-1} = h_i$.

\begin{equation}
    p^\prime_i(t_i) = \frac{ z_i \cosh(\tau_i h_i) -z_{i-1} }{\tau_i \sinh(\tau_i h_i)}
        + \frac{y_i - y_{i-1} +  (z_{i-1} - z_i)/\tau_i^2}{h_i}
\end{equation}

\begin{equation}
    p^\prime_i(t_{i-1}) = \frac{ z_i - z_{i-1} \cosh(\tau_i h_i) }{\tau_i \sinh(\tau_i h_i)}
        + \frac{y_i - y_{i-1} +  (z_{i-1} - z_i)/\tau_i^2}{h_i}
\end{equation}

Substituting these in:

\begin{multline}
    \int_{t_{i - 1}}^{t_i}p^{\prime\prime}_i(t)^2 + \tau_i^2 p^{\prime}(t)^2 dt
    = z_i\biggl( \frac{ z_i \cosh(\tau_i h_i) -z_{i-1} }{\tau_i \sinh(\tau_i h_i)}
    + \frac{y_i - y_{i-1} +  (z_{i-1} - z_i)/\tau_i^2}{h_i} \biggr) \\
    - z_{i-1} \biggl(\frac{ z_i - z_{i-1} \cosh(\tau_i h_i) }{\tau_i \sinh(\tau_i h_i)}
    + \frac{y_i - y_{i-1} +  (z_{i-1} - z_i)/\tau_i^2}{h_i} \biggr) \\
    + \frac{(y_i - y_{i-1})^2\tau_i^2 + (y_i - y_{i-1})(z_{i-1} - z_i)}{h_i} 
\end{multline}

Rearranging.

\begin{multline}
    = z_i\biggl( \frac{ z_i \cosh(\tau_i h_i) -z_{i-1} }{\tau_i \sinh(\tau_i h_i)} \biggr)
    - z_{i-1} \biggl(\frac{ z_i - z_{i-1} \cosh(\tau_i h_i) }{\tau_i \sinh(\tau_i h_i)} \biggr) \\
    - \frac{(y_i - y_{i-1})(z_{i-1} - z_i) +  (z_{i-1} - z_i)^2/\tau_i^2}{h_i}  \\
    + \frac{(y_i - y_{i-1})^2\tau_i^2 + (y_i - y_{i-1})(z_{i-1} - z_i)}{h_i} 
\end{multline}

Cancelling terms.

\begin{multline}
    = z_i\biggl( \frac{ z_i \cosh(\tau_i h_i) -z_{i-1} }{\tau_i \sinh(\tau_i h_i)} \biggr)
    - z_{i-1} \biggl(\frac{ z_i - z_{i-1} \cosh(\tau_i h_i) }{\tau_i \sinh(\tau_i h_i)} \biggr) \\
    + \frac{(y_i - y_{i-1})^2\tau_i^2 - (z_{i-1} - z_i)^2/\tau_i^2 }{h_i} 
\end{multline}

Rearranging.

\begin{multline}
    = (z_i^2 + z_{i-1}^2) \frac{\cosh(\tau_i h_i)}{\tau_i \sinh(\tau_i h_i)}
    -z_i z_{i-1} \frac{2}{\tau_i \sinh(\tau_i h_i)} \\
    + \frac{(y_i - y_{i-1})^2\tau_i^2 - (z_{i-1} - z_i)^2/\tau_i^2 }{h_i} 
\end{multline}

Multiplying out the squared differences.

\begin{multline}
    = (z_i^2 + z_{i-1}^2) \frac{\cosh(\tau_i h_i)}{\tau_i \sinh(\tau_i h_i)}
    -z_i z_{i-1} \frac{2}{\tau_i \sinh(\tau_i h_i)} \\
    + \frac{(y_i^2 + y_{i-1}^2 -2 y_i y_{i-1} )\tau_i^2 - (z_i^2 + z_{i-1}^2 - 2 z_i z_{i-1} )/\tau_i^2 }{h_i} 
\end{multline}

Rearranging again so it can be seen that this is a quadratic form.

\begin{multline}
    = z_i^2 \biggl( \frac{\cosh(\tau_i h_i)}{\tau_i \sinh(\tau_i h_i)} -
    \frac{1}{\tau_i^2 h_i} \biggr) 
    + z_{i-1}^2 \biggl( \frac{\cosh(\tau_i h_i)}{\tau_i \sinh(\tau_i h_i)} -
    \frac{1}{\tau_i^2 h_i} \biggr) \\
    + y_i^2 \frac{\tau_i^2}{h_i} + y_{i-1}^2 \frac{\tau_i^2}{h_i} 
    + z_i z_{i-1} 2 \biggl( \frac{1}{\tau_i^2 h_i} - \frac{1}{\tau_i \sinh(\tau_i h_i)} \biggr) \\
    - y_i y_{i-1} \frac{2 \tau_i^2}{h_i}
\end{multline}

\end{document}